%% LyX 2.0.3 created this file.  For more info, see http://www.lyx.org/.
%% Do not edit unless you really know what you are doing.
\documentclass[english,annual]{acmsiggraph}
\usepackage[T1]{fontenc}
\usepackage[latin9]{inputenc}
\usepackage{url}
\usepackage{graphicx}

\makeatletter
%%%%%%%%%%%%%%%%%%%%%%%%%%%%%% User specified LaTeX commands.
% for the following cases use the listed document class option:
% [annual] - Technical paper accepted for presentation at the ACM SIGGRAPH 
%   or SIGGRAPH Asia annual conference.
% [sponsored] - Short or full-length technical paper accepted for 
%   presentation at an event sponsored by ACM SIGGRAPH
%   (but not the annual conference Technical Papers program).
% [abstract] - A one-page abstract of your accepted content
%   (Technical Sketches, Posters, Emerging Technologies, etc.). 
%   Content greater than one page in length should use the "[sponsored]"
%   parameter.
% [preprint] - A preprint version of your final content.
% [review] - A technical paper submitted for review. Includes line
%   numbers and anonymization of author and affiliation information.

% When you submit your paper for review, please use the \TOGonlineID''
% command to include the online ID value assigned to your paper by the
% submission management system. Replace '45678' with the value you were
% assigned.
\TOGonlineid{45678}

% If you are preparing a preprint of your accepted paper, and your paper
% will be published in an issue of the ACM "Transactions on Graphics''
% journal, replace the "0'' values in the commands below with the correct
% volume and number values for that issue - you'll get them before your
% final paper is due.
\TOGvolume{0}
\TOGnumber{0}

% The TOGarticleDOI' command accepts the DOI information provided to you
% during production, and which makes up the URLs which identifies the ACM
% article page and direct PDF link in the ACM Digital Library.
% Replace "1111111.2222222'' with the values you are given.
% \TOGarticleDOI{1111111.2222222}

% If you would like to include links to personal repositories for auxiliary
% material related your research contribution, you may use one or more of
% these commands to define an appropriate URL. The "\TOGlinkslist'' command
% found just before the first section of your document will add hyperlinked
% icons to your document, in addition to hyperlinked icons which point to
% the ACM Digital Library article page and the ACM Digital Library-held PDF.
% \TOGprojectURL{}
% \TOGvideoURL{}
% \TOGdataURL{}
% \TOGcodeURL{}

% Paper title.
\title{Real-Time Rendering and 3D Games Programming: Custom Rendering Engine}

% Author and Affiliation (single author).
%\author{Name \thanks{e-mail: name@unknown.uu}\\ Research Institute}

% Author and Affiliation (multiple authors).
\author{S�ren V. Poulsen\thanks{e-mail: svipo@dtu.dk}\\M.Sc. Stud.\\DTU Compute, Lyngby %
\and Geoff Leach\thanks{e-mail:gl@rmit.edu.au}\\Lecturer and Coordinator\\RMIT University%
\and Tim Mutton\thanks{e-mail:tim.mutton@rmit.edu.au}\\Head Tutor\\RMIT University}

% The ``pdfauthor'' command accepts the authors of the work,
% comma-delimited, and adds this information to the PDF metadata.
\pdfauthor{S�ren V. Poulsen}

% Keywords that describe your work.
\keywords{opengl,rendering,}

\@ifundefined{showcaptionsetup}{}{%
 \PassOptionsToPackage{caption=false}{subfig}}
\usepackage{subfig}
\makeatother

\usepackage{babel}
\begin{document}

\teaser{\includegraphics[width=0.2\paperwidth]{\string"images/RMIT Portal_008\string".png}\,\includegraphics[width=0.2\paperwidth]{\string"images/RMIT Portal_009\string".png}\,\includegraphics[width=0.2\paperwidth]{\string"images/RMIT Portal_010\string".png}\,\includegraphics[width=0.2\paperwidth]{\string"images/RMIT Portal_012\string".png}\caption{Crytek Sponza scene rendering stages. Subfigures from left to right:
full shading mode, depth buffer, normal buffer, and ambient occlusion.}
}

\maketitle
\maketitle 
\begin{abstract}
This paper describes the design and implementation of a custom engine
for rendering various 3D scenes using OpenGL. Using modern rendering
techniques, features such as dynamic point lights, screen-space ambient
occlusion and deferred rendering has been implemented in a GNU/Linux
C++11 application with minimal use of external libraries. The project
has been completed as an assignment in the course \emph{COSC1224/1226
Real-Time Rendering and 3D Games Programming}%
\footnote{\url{http://goanna.cs.rmit.edu.au/~gl/teaching/rtr&3dgp/}%
}\emph{ }coordinated by professor Geoff Leach of RMIT University\emph{. }

The project has been completed in 12 weeks during semester 2 of 2013
by M.Sc. Stud. S�ren V. Poulsen of DTU Compute in Lyngby, Denmark.
\end{abstract}

\date{October 2013}


\section{Introduction}

%\copyrightspace

The purpose of this project assignment has been to demonstrate skills
and knowledge of various OpenGL features. Initially the project schedule
included portal rendering capabilities, however this has not been
implemented. Instead, more advanced rendering features such as deferred
rendering was chosen.

Full source code and assets are available via GitHub%
\footnote{\url{https://github.com/vinther/rmit-rendering}%
}.


\section{Features}

Below is a list of the features that are currently part of the rendering
engine in a working state:
\begin{itemize}
\item Fully deferred rendering 
\item Color and normal displacement textures
\item Dynamic and unlimited point light sources
\item Screen-space ambient occlusion
\item Efficient CPU-based ray/scene intersection test using bounding volume
hierarchies
\item Crytek Sponza scene, 262k triangles, 256 lights: 40-60 (capped) frames
per second on Nvidia 650m hardware via bumblebee (Optimus hardware
switch emulator) on Linux Mint 15 ``olivia''.
\end{itemize}

\section{Implementation}

The application is based on the SDL2 library which provides the OpenGL
context and window management. Other libraries such as \emph{GLEW}
and \emph{libjpeg} has been used to provide math functions and image
loading capabilities, respectively. The \emph{Open Asset Import Library}
(\emph{assimp}) is used to load scenes from Wavefront OBJ files. Several
C++0x/C++11 features such as smart pointers and range-based \emph{for}
loops are heavily utilized throughout the implementation.


\subsection{Deferred Passes}

\begin{figure}
\begin{centering}
\subfloat[Bump-mapping disabled]{\includegraphics[width=0.45\columnwidth]{\string"images/RMIT Portal_021\string".png}

}\subfloat[Bump-mapping enabled]{\includegraphics[width=0.45\columnwidth]{\string"images/RMIT Portal_020\string".png}

}
\par\end{centering}

\caption{Surface normals of lions head\label{fig:Surface-normals-of}}
\end{figure}


Deferred rendering is a rendering technique that separates geometry
and lighting operations into multiple passes. Using this technique
has the great benefit of reducing the performance hit of having many
lights in a scene regardless of the complexity of the scene geometry.
However, this technique complicates having translucent materials (something
which is not handled at all in this implementation) since each pixel
can only contain information about a single, opaque surface.

In the first pass (the geometry or G-pass), information about the
scene geometry is rendered into multiple buffers (which in turn are
simple textures). In this application the texture color (albedo) is
stored in a $4\times8\,\mbox{bit}$ texture buffer, the surface normal
in a packed $2\times16\,\mbox{bit}$ floating point (FP) texture buffer,
and finally the depth and stencil in a $24+8\,\mbox{bit}$ FP/integer
texture buffer. The normal maps are displaced using height map displacement
(bump-mapping) in order to increase details in lighting calculations
without having to increase the complexity of the scene geometry (fig.
\ref{fig:Surface-normals-of}).


\subsection{Lighting}

\begin{figure}
\begin{centering}
\subfloat[SSAO disabled]{\includegraphics[width=0.45\columnwidth]{\string"images/RMIT Portal_023\string".png}

}\subfloat[SSAO enabled]{\includegraphics[width=0.45\columnwidth]{\string"images/RMIT Portal_024\string".png}

}
\par\end{centering}

\caption{Display of SSAO effect (bump-mapping enabled)\label{fig:Display-of-SSAO}}
\end{figure}


Lighting happens in a separate, deferred stage. This implementation
supports a single ambient light and a potentially infinite number
of spherical point lights.

To determine the ambient light contribution a single full-screen quadrangle
is rendered to the light accumulation buffer. The quadrangle is texture
with the contents of the geometry texture color (from the geometry
buffer) multiplied by some constant in the unit interval. An ambient
occlusion term is multiplied onto the ambient light contribution (fig.
\ref{fig:Display-of-SSAO}) using a hemispherical approach%
\footnote{\url{http://blog.evoserv.at/index.php/2012/12/hemispherical-screen-space-ambient-occlusion-ssao-for-deferred-renderers-using-openglglsl/}%
}. This term provides a subtle shadow along corners and creases that
under physically correct circumstances are likely to receive less
contribution from ambient lighting.

Point light source contributions are determined by rendering spheres
with radii corresponding to the intensity of the light (fig. \ref{fig:Point-light-sphere}).
When doing the per-pixel (fragment) shading of the rendered spheres,
the geometry properties of each pixel (i.e. depth and normals) are
looked up in the geometry buffer. Pixels outside the area of influence
contributes no light (fig. \ref{fig:Influenced-surfaces}) while the
remaining contribute light according to their distance from the light
(fig. \ref{fig:Attenuated-light-(linear)}), surface texture color
and cosine weighting (fig. \ref{fig:Final-contribution}). 

\begin{figure}
\begin{centering}
\subfloat[Point light sphere backfaces\label{fig:Point-light-sphere}]{\includegraphics[width=0.45\columnwidth]{\string"images/RMIT Portal_015\string".png}



}\subfloat[Influenced surfaces\label{fig:Influenced-surfaces}]{\includegraphics[width=0.45\columnwidth]{\string"images/RMIT Portal_016\string".png}

}
\par\end{centering}

\begin{centering}
\subfloat[Attenuated light (linear)\label{fig:Attenuated-light-(linear)}]{\includegraphics[width=0.45\columnwidth]{\string"images/RMIT Portal_017\string".png}

}\subfloat[Final contribution\label{fig:Final-contribution}]{\includegraphics[width=0.45\columnwidth]{\string"images/RMIT Portal_019\string".png}

}
\par\end{centering}

\caption{Point light example}


\end{figure}



\subsection{Ray-tracing}

In an effort to support basic game-like features such as shooting
weapons, the application is capable of doing ray-scene intersection.
That is, given a starting point and direction, determine the closest
point (if any) of intersection with the triangle meshes that makes
up the scene. A typical scene has many triangles, so a naive $O\left(n\right)$
search is deemed to be too inefficient when scenes become non-trivial.

Instead a bounding volume hierarchy (BVH) of axis-aligned bounding
boxes (AABBs) containing triangles is built in the initialization
stage of the engine. The octree data structure has been selected as
the BVH structure. When querying for ray-triangle intersections, the
top level (root) AABB of the BVH tree is tested for intersection with
the given ray. If there is such an intersection, the 8 children AABBs
beneath the root is tested and so forth until only leaf nodes are
left. The leaf nodes contains simple triangle lists, which are then
tested against the given ray.

In average- and best-case scenarios this method significantly reduces
ray-triangle intersection tests and makes even large amounts of ray-scene
queries for each frame feasible.


\subsection{Asset Handling}

Any rendering or game engine requires handling of assets such as shader
programs, textures, models, scripts, etc. The efficiency of the asset
handling mechanics is a determining factor for loading times and/or
memory usage, which makes it an important and essential part of any
rendering engine.

This application uses a simple string-based caching mechanic. Each
asset is given a name, which is stored as a simple textual string,
that in most cases is directly related to the file path of the asset.
 When an asset is requested for usage (e.g. a model or texture file)
the name of the requested asset is looked up in the asset cache. If
found, a reference to the already loaded asset is returned, leaving
the file system untouched. If the name is not found the asset is loaded
from the file system and entered into the cache system.

Having all assets handled by the same asset management system has
other benefits besides reduced file system utilization. For example,
this application features hot-loadable assets meaning that assets
that are altered by an external application are almost instantaneously
reloaded in the engine. This has been accomplished using the Linux
kernel subsystem \emph{inotify} and has proven to be very beneficial
while debugging the application. 


\section{Conclusion}

Despite not achieving a working portal rendering engine as originally
intended, several advanced rendering techniques have been implemented
together with other technically challenging features. Many trivial
(and non-trivial) optimizations are not part of the current implementation
thus making it an ideal target for further development. 

\bibliographystyle{acmsiggraph}
\bibliography{../examples/biblioExample}
 
\end{document}
